\PassOptionsToPackage{unicode=true}{hyperref} % options for packages loaded elsewhere
\PassOptionsToPackage{hyphens}{url}
%
\documentclass[]{article}
\usepackage{lmodern}
\usepackage{amssymb,amsmath}
\usepackage{ifxetex,ifluatex}
\usepackage{fixltx2e} % provides \textsubscript
\ifnum 0\ifxetex 1\fi\ifluatex 1\fi=0 % if pdftex
  \usepackage[T1]{fontenc}
  \usepackage[utf8]{inputenc}
  \usepackage{textcomp} % provides euro and other symbols
\else % if luatex or xelatex
  \usepackage{unicode-math}
  \defaultfontfeatures{Ligatures=TeX,Scale=MatchLowercase}
\fi
% use upquote if available, for straight quotes in verbatim environments
\IfFileExists{upquote.sty}{\usepackage{upquote}}{}
% use microtype if available
\IfFileExists{microtype.sty}{%
\usepackage[]{microtype}
\UseMicrotypeSet[protrusion]{basicmath} % disable protrusion for tt fonts
}{}
\IfFileExists{parskip.sty}{%
\usepackage{parskip}
}{% else
\setlength{\parindent}{0pt}
\setlength{\parskip}{6pt plus 2pt minus 1pt}
}
\usepackage{hyperref}
\hypersetup{
            pdftitle={Práctica para parcial},
            pdfborder={0 0 0},
            breaklinks=true}
\urlstyle{same}  % don't use monospace font for urls
\setlength{\emergencystretch}{3em}  % prevent overfull lines
\providecommand{\tightlist}{%
  \setlength{\itemsep}{0pt}\setlength{\parskip}{0pt}}
\setcounter{secnumdepth}{0}
% Redefines (sub)paragraphs to behave more like sections
\ifx\paragraph\undefined\else
\let\oldparagraph\paragraph
\renewcommand{\paragraph}[1]{\oldparagraph{#1}\mbox{}}
\fi
\ifx\subparagraph\undefined\else
\let\oldsubparagraph\subparagraph
\renewcommand{\subparagraph}[1]{\oldsubparagraph{#1}\mbox{}}
\fi

% set default figure placement to htbp
\makeatletter
\def\fps@figure{htbp}
\makeatother


\title{Práctica para parcial}
\date{}

\begin{document}
\maketitle

\hypertarget{dataset-nycflights13}{%
\section{Dataset nycflights13}\label{dataset-nycflights13}}

\begin{enumerate}
\def\labelenumi{\arabic{enumi}.}
\tightlist
\item
  Instalar el paquete ``nycflights13'' y cargarlo en la sesión.
\item
  El paquete contiene un objeto de datos llamados ``flights''. Ver el
  help de flights para ver de que se tratan.
\item
  Calcular cuantos vuelos salen de NYC por dia. Hacerlo de dos maneras:
  una con summarise() + n() y otra con tally(). Pensar de antemano las
  dimensiones del dataframe que van a generar y chequear las dimensiones
  después de generarlo.
\item
  Agregarle a este último dataframe una columna con el día de la semana.
\item
  Hacer un boxplot, para cada día de la semana, de los vuelos por dia.
  Tip: le paso estas dos variables al aes() de la capa boxplot. Cuál es
  el día menos similar al resto?
\item
  Qué dia de la semana tiene mayor demora en las salidas de los vuelos?
\item
  Para qué destino hay mayor demora en los vuelos de salida?
\end{enumerate}

\end{document}
