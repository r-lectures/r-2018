% Created 2018-05-16 Wed 12:46
% Intended LaTeX compiler: pdflatex
\documentclass[11pt]{article}
\usepackage[utf8]{inputenc}
\usepackage[T1]{fontenc}
\usepackage{graphicx}
\usepackage{grffile}
\usepackage{longtable}
\usepackage{wrapfig}
\usepackage{rotating}
\usepackage[normalem]{ulem}
\usepackage{amsmath}
\usepackage{textcomp}
\usepackage{amssymb}
\usepackage{capt-of}
\usepackage{hyperref}
\date{16/05/2017}
\title{R lectures}
\hypersetup{
 pdfauthor={Luis G. Moyano},
 pdftitle={R lectures},
 pdfkeywords={},
 pdfsubject={},
 pdfcreator={Emacs 25.3.1 (Org mode 9.1.5)}, 
 pdflang={English}}
\begin{document}

\maketitle
\begin{verbatim}
--- 
layout: default 
title: Clase 11
--- 
\end{verbatim}
\section*{Repaso de la clase anterior}
\label{sec:org2bb1966}
\subsection*{Funciones matemáticas básicas}
\label{sec:org3f9ae73}
\begin{verbatim}
- exp(): función exponencial, base e
- log(): logaritmo natural 
- log10(): logaritmo base 10
- sqrt(): raiz cuadrada
- abs(): valor absoluto
- sin(), cos(), etc.: funciones trigonométricas
- min(),  max(): valor mínimo y máximo de un vector
- which.min() and which.max(): índice del valor mínimo y máximo 
- pmin() and pmax(): mínimos y máximos para varios vectores, por elemento
- sum() and prod(): suma y producto de elementos de vectores
- cumsum() and cumprod(): suma acumulada y producto acumulado de elementos de vectores
- round(), floor(), and ceiling(): redondeo al entero más próximo, al menor o al mayor, respectivamente
- factorial(): función factorial
- %%: operador módulo y %/%: operador división por enteros
\end{verbatim}
\subsection*{Cálculo}
\label{sec:org8ffdd30}
\begin{enumerate}
\item R tiene capacidades tanto para hacer cálculos numéricos como analíticos:
\item Existen paquetes para ecuaciones diferenciales (\texttt{odesolve}), y para extender la capacidad simbólica usando el sistema Yacas (\texttt{ryacas}). Ver CRAN.
\end{enumerate}
\subsection*{Álgebra Lineal}
\label{sec:org4589db8}
\begin{enumerate}
\item Multiplicación de matrices - operador \%*\%
\item Sistema de ecuaciones lineales (o invertir una matriz) con \texttt{solve()}
\item Autovectores y autovalores - \texttt{eigen()}
\item Otras operaciones posibles \texttt{t()}, \texttt{svd()}, etc.
\end{enumerate}

\subsection*{Otros paquetes de interés}
\label{sec:orgf718d0d}
\begin{itemize}
\item \href{https://cran.r-project.org/web/views/NumericalMathematics.html}{Numerical Mathematics}
\item \href{https://cran.r-project.org/web/views/DifferentialEquations.html}{Ecuaciones diferenciales}
\begin{itemize}
\item tienen también el libro "2012 - Book - Solving Differential Equations in R.pdf" en \#bibliografia @slack
\end{itemize}
\item \href{https://cran.r-project.org/web/views/TimeSeries.html}{Series temporales}
\item \href{https://cran.r-project.org/web/views/Optimization.html}{Optimización y programación matemática}
\item Aritmética de precisión múltiple con \href{https://cran.r-project.org/web/packages/gmp/index.html}{gmp}
\item Paquete \href{https://cran.r-project.org/web/packages/gsl/index.html}{gsl}, una interface a la Biblioteca Científica GNU
\item Mil cosas más :)
\end{itemize}

\section*{Estadística}
\label{sec:orgb4b58a2}
\subsection*{Estadística descriptiva}
\label{sec:org6e1dffb}
Una manera de hacer estadística descriptiva es con \texttt{sapply}:
\begin{verbatim}
sapply(mydata, mean, na.rm=TRUE) 
\end{verbatim}
(podemos usar \texttt{mean}, \texttt{sd}, \texttt{var}, \texttt{min}, \texttt{max}, \texttt{median}, \texttt{range}, y \texttt{quantile}). O \texttt{summary()}.

Tenemos también \texttt{group\_by() + summarise()} con las mismas funciones básicas.
\subsection*{Distribuciones}
\label{sec:orgc7d3c72}
En general R usa la siguiente convención:

\begin{itemize}
\item \texttt{dDIST(x, ...)} es la función distribución de probabilidad (PDF). Devuelve la prob. de observar un
valor x
\item \texttt{pDIST(x, ...)} es la función cumulativa de probabilidad (CDF). Devuelve la prob. de obervar un
valor menor a x (mayor si usamos lower.tail=F)
\item \texttt{rDIST(n, ...)} es un generador de números aleatorios que devuelve n valores sacados de una distr. DIST.
\item \texttt{qDIST(p, ...)} es la función cuartil que devielve el x que corresponde al percentil p de DIST. Si
lower.tail=F, devuelve 1 - el percentil p.
\end{itemize}

Para ver las distribuciones disponibles, ver el \href{https://cran.r-project.org/web/views/Distributions.html}{task view de distribuciones}
\subsubsection*{Distribución Normal}
\label{sec:org44b0d8d}
Construyo una figura de la función normal usando un vector entre -5 y 5 con 100 puntos.

\begin{verbatim}
library(ggplot2)
set.seed(8888) ## elijo la semilla para poder "controlar" la aleatoridad
x <- seq(from=-5, to=5, length.out=100) # el intervalo [-5 5]
f <- dnorm(x) # normal con media 0 y sd 1 
ggplot(data.frame(col1=x, col2=f), aes(x=col1, y=col2)) + geom_line()
\end{verbatim}
\subsubsection*{Otras distribuciones}
\label{sec:org32c5a75}
Construyo un vector de 10\^{}5 puntos que contenga valores estocásticos extraidos de una dist. Binomial
de \texttt{n=5} (número de intentos) y \texttt{p=0.5} (probabilidad de éxito).

\begin{verbatim}
x <- rbinom(100000,5,0.5)
mean(x)
# [1] 2.5004

mean(x >= 4)
# [1] 0.18829
\end{verbatim}
\section*{Modelado estadístico}
\label{sec:org2164d47}
Modelado se refiere a proponer determinadas relaciones entre variables, típicamente cuál es la
relación entre una variable dependiente o \emph{variable respuesta} y otras variables independientes o
\emph{variables explicativas}. 

En R la función lm() se usa para regresión lineal (\emph{linear models}) y glm() para \emph{generalized linear models}.
\subsubsection*{Regresión lineal (lm)}
\label{sec:orgb3bc457}
Construimos un "modelo" (una relación) entre variables dependientes e independientes optimizando
parámetros para poder predecir.

\begin{itemize}
\item 1. Propongo una determinada relación de variables
\item 2. Calculo coeficientes del modelo
\item 3. Compruebo que tan bien se ajusta el modelo a nuevas observaciones
\end{itemize}

\begin{verbatim}
y[i] ~ f(x[i,]) = b[1] x[i,1] + ... b[n] x[i,n]
## b[i] son los coeficientes o betas
\end{verbatim}

\subsubsection*{Ejemplo con datos de 2011 US Census PUMS}
\label{sec:org3a5c616}

Pueden bajar los datos de \href{https://github.com/WinVector/zmPDSwR/raw/master/PUMS/psub.RData}{acá}.

\begin{verbatim}
## hacemos la regresión:
load("psub.RData")
dtrain <- subset(psub, ORIGRANDGROUP >= 500)
dtest  <- subset(psub, ORIGRANDGROUP < 500)
model  <- lm(log(PINCP,base=10) ~ AGEP + SEX + COW + SCHL, data=dtrain) 
dtest$predLogPINCP <- predict(model,newdata=dtest) 

## resultados:
summary(model)

## graficamos:
library(ggplot2)
ggplot(data=dtest,aes(x=predLogPINCP,y=log(PINCP,base=10))) + geom_point(alpha=0.2,color="black") + 
geom_smooth(aes(x=predLogPINCP, y=log(PINCP,base=10)),color="black") +
geom_line(aes(x=log(PINCP,base=10), y=log(PINCP,base=10)),color="blue",linetype=2) +
scale_x_continuous(limits=c(4,5)) +
scale_y_continuous(limits=c(3.5,5.5))

## residuos:
ggplot(data=dtest,aes(x=predLogPINCP, y=predLogPINCP-log(PINCP,base=10))) +
geom_point(alpha=0.2,color="black") +
geom_smooth(aes(x=predLogPINCP, y=predLogPINCP-log(PINCP,base=10)), color="black")

\end{verbatim}
\subsubsection*{Regresión lineal generalizada (glm)}
\label{sec:org15278a8}

Los modelos lineales asumen que el valor predicho es continuo y que los errores van a ser
"normales". Los modelos lineales generalizados relajan estas suposiciones.

\begin{verbatim}
## expresión general
glm(formula, family=familytype(link=linkfunction), data=)
\end{verbatim}

Ejemplito: Regresión logística, para variables categóricas.

\begin{verbatim}
# F es un factor binario
# x1, x2 y x3 son predictores continuos 
fit <- glm(F~x1+x2+x3,data=mydata,family=binomial())
summary(fit) # resultados
exp(coef(fit)) # coeficientes
predict(fit, type="response") # predicciones
residuals(fit, type="deviance") # residuos 

\end{verbatim}
\section*{Estadística avanzada - material infinito}
\label{sec:org2f122af}
\begin{itemize}
\item \href{https://stat.ethz.ch/R-manual/R-devel/library/stats/html/00Index.html}{Paquete stats}
\item \href{https://cran.r-project.org/web/views/Distributions.html}{CRAN view de distribuciones}
\item \href{https://www.stats.ox.ac.uk/pub/MASS4/}{Modern Applied Statistics with S. Fourth Edition} - \href{https://cran.r-project.org/web/packages/MASS/index.html}{(MASS book)}
\item \href{http://statweb.stanford.edu/\~tibs/ElemStatLearn/}{The elements of statistical learning} - \href{https://cran.r-project.org/web/packages/ElemStatLearn/index.html}{(ElemStatLearn book)}
\end{itemize}
\section*{Práctica 11 bis}
\label{sec:org32d71c2}
\begin{enumerate}
\item Generar un conjunto de 10\^{}3 números aleatorios sacados de una distribución lognormal con promedio
5 y variación estándar 1. Hacer su histograma con ggplot2 y compararlo con la lognormal con
parámetros (5,1) en la misma figura. Generar otras 10\^{}6 números y agregar su histograma a la
figura (o sea, que queden dos histogramas y una curva). Usar el argumento \emph{alpha} para poder
distinguir los histogramas.
\end{enumerate}

\subsection*{Práctica 11 bis}
\label{sec:orgda154cb}
\begin{enumerate}
\item i)  Bajarse el Quijote de \href{http://www.gutenberg.org}{"Project Gutenberg"} en formato texto.

ii) Meter el libro en un vector de tipo "character", una palabra en cada elemento del
vector. Tip: \texttt{stringr::str\_split()}. Cuántas palabras hay en total? Cuántas únicas?

iii) Cuántas veces aparece cada palabra? Tip: \texttt{dplyr} + nombrar las columnas + \texttt{n()} o
\texttt{tally()}. Cuál es la 1era palabra del ranking? Cuanto aparece "Quijote" y en que ranking?

iv) Hacer un plot del ranking vs. su frecuencia. Qué llama la atención? Tip: ambos ejes logarítmicos.

v) CERRQué función de distribución describiría bien lo que encontramos? Puede ser una normal? Probar
distintas distribuciones (con \texttt{dDIST()}) para aproximar los datos. Tip: poner "quijote
distribution" en Google Scholar. Qué parámetro que controla la distribución es importante y
cuando vale (intentar calcularlo con R)?
\end{enumerate}
\subsection*{Práctica 11 tris}
\label{sec:org6b777cf}
\begin{enumerate}
\item En los datos de diamantes, hacer una regresión lineal de la variable logaritmo del precio como
función del logaritmo del peso (carat). Sacar los coeficientes y usarlos para graficar el modelo (con
una línea) sobre el scatterplot (usar geom\_hex() para este último). Luego graficar los residuos
en otro gráfico.
\item Si tengo dos vectores a y b del mismo largo, que obtengo al hacer sum(a * b)? Y sqrt(sum(a * a))?
\item En un dia de sol, hay dos mesas en un jardín inglés. En cada mesa hay algunos pájaros,
tranquis. Uno de la primer mesa les dice a los de la segunda: "si se viene uno de uds. acá,
entonces vamos a ser la misma cantidad en las dos mesas". "Si", le responden, "pero si se viene
uno de uds. para acá, vamos a ser el doble acá que la de ustedes". Escriban unas ecuaciones para
resolver en R y saber cuántos pájaros había en cada mesa. (Tomado de "Linear algebra in R", Søren Højsgaard
15 de Febrero de 2005.)
\end{enumerate}
\end{document}